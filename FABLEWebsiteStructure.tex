\documentclass[10pt]{article}
\usepackage{tabulary}
\usepackage{empheq}
\usepackage[margin=1.0in]{geometry} 
\usepackage{amsmath, mathtools}
\usepackage{parskip}
\providecommand{\e}[1]{\ensuremath{\times 10^{#1}}}
\newcommand{\dx}[1]{\mathrm{d}{#1}}
\usepackage{color}
\newcommand{\hilight}[1]{\colorbox{yellow}{#1}}


\begin{document}

\title{FABLE Documentation for RDCEP Website}
\maketitle

%%%%%%%%%%%%%%%%%%%%%%%%%%%%%%%%%%%%%%%%%%%%%%%%%%%%%%%%%%
%%%%%%%%%%%%%%%%%%%%%%%%%%%%%%%%%%%%%%%%%%%%%%%%%%%%%%%%%%

\section{Home Page}
\paragraph{Layout}
This is the first section of the page. The image of the Allocation of Land graph should be to the right of the introduction. We should move the "Limitations" section to the documentation tab. 
\subsection{Introduction}
Competition for land use over time is projected to increase. Changes in demand for energy, forestry products, and food are driven by expected increases in population and changing diets over time, especially in the developing world. These issues are compounded by the uncertainty of agricultural and forestry productivity due to climate change. As such, research has begun to focus more intensely on the optimal allocation of land across sectors and time.\\
FABLE differs from other land use models by integrating components of interest from several research communities. It seeks to address the gap between two land use models. The first are indirect land use change models (iLUCs) that assess short term costs of substituting biofuel crops in areas of highly productive agriculture, therefore forcing food crops to less productive areas or into carbon-rich natural environments. The second are so called integrated assessment models (IAMs) that account for significant greenhouse gas reductions by substituting biofuel production for conventional energy demands.

\subsection{Outline of Model}

The Forest, Agriculture, and Biofuels in a Land use model with Environmental services (FABLE) is a global, intertemporally-consistent optimization model for land use analysis. The model incorporates projections of demands for food, energy and bioenergy, timber and recreation or environmental services in addition to greenhouse gas mitigation targets to solve a constrained welfare maximization problem. The model runs over the period 2005 - 2204, and has accurately predicted developments over the existing history. \\
FABLE takes income, population, wages, oil prices, total factor productivity and other input prices as exogenous. The model is solved by maximizing discounted social welfare across 200 years of global demand for land associated with crop and pasture demand, as well as energy (via conventional liquid fuels and biofuel substitutes), timber and recreational services. \\
To learn more please see the \hilight{Documentation} page.


%%%%%%%%%%%%%%%%%%%%%%%%%%%%%%%%%%%%%%%%%%%%%%%%%%%%%%%%%%
%%%%%%%%%%%%%%%%%%%%%%%%%%%%%%%%%%%%%%%%%%%%%%%%%%%%%%%%%%

\section{FABLE Baseline Tab}
\paragraph{Layout}
This tab has the sectors of the model listed to the left of the page (all sector headings should be links to the documentation page for people to learn more about the model, and formatted as they appear in the documentation page). For the main part of the page, lay out the four main FABLE graphs with spaces underneath for captions. Then have sections underneath the graphs for discussion. 

\subsection{Graphs \& Captions}
Graphs to include:
\begin{itemize}
\item Allocation of Land (CAPTION)
\item Land Based GHG Emissions (CAPTION)
\item Land Based Goods and Services (CAPTION)
\item Consumption of Biofuels (CAPTION)
\end{itemize}

\subsection{Model Discussion}

TO INCLUDE




%%%%%%%%%%%%%%%%%%%%%%%%%%%%%%%%%%%%%%%%%%%%%%%%%%%%%%%%%%
%%%%%%%%%%%%%%%%%%%%%%%%%%%%%%%%%%%%%%%%%%%%%%%%%%%%%%%%%%

\section{Run FABLE Tab}
\paragraph{Layout}
Have the same model sectors on the left of the page, but with places underneath each sector for variable input (suggestions listed below). We'll have the plots (with time slider capability), but for now let's leave this TBD. 


%%%%%%%%%%%%%%%%%%%%%%%%%%%%%%%%%%%%%%%%%%%%%%%%%%%%%%%%%%
%%%%%%%%%%%%%%%%%%%%%%%%%%%%%%%%%%%%%%%%%%%%%%%%%%%%%%%%%%

\section{Glossary Tab}
\paragraph{Layout}
Have model sectors on the left (still same style as in the documentation) with links to the Glossary headings for each sector. The main part of the page has the list of glossary topics, organized by sector. 

\subsection{Glossary}
\paragraph{Resource Use - Land}
\begin{itemize}
\item Protected Land - natural parks, biodiversity reserves, and other types of protected forests used to produce ecosystem services
\end{itemize}
\paragraph{Resource Use - Fossil Fuels}
\paragraph{Agrochemical}
\paragraph{Agricultural}
\begin{itemize}
\item CES Function
\item Imperfect Substitutes
\end{itemize}
\paragraph{Livestock Farming}
\paragraph{Food Processing}
\paragraph{Biofuels}
\begin{itemize}
\item 1st Generation Biofuel - ie corn or sugarcane ethanol
\item 2nd Generation Biofuel - ie cellulosic biomass to liquid diesel fuels
\end{itemize}
\paragraph{Energy}
\paragraph{Forestry}
\begin{itemize}
\item Vintage - cohort of trees of same age
\end{itemize}
\paragraph{Timber}
\paragraph{Ecosystem Services}
\paragraph{Other Goods and Services}
\paragraph{Greenhouse Gas Emissions}
\paragraph{Preferences}
\paragraph{Welfare}



%%%%%%%%%%%%%%%%%%%%%%%%%%%%%%%%%%%%%%%%%%%%%%%%%%%%%%%%%%
%%%%%%%%%%%%%%%%%%%%%%%%%%%%%%%%%%%%%%%%%%%%%%%%%%%%%%%%%%

\section{Documentation Tab}
\paragraph{Layout}
Left hand side has a list of links to the sectors of the model, plus a "Math behind the model" and "Resources" link. All of these should link to the main body of the page for content. 

\subsection{Sectors}

\paragraph{Resource Use - Land}
\paragraph{Resource Use - Fossil Fuels}
\paragraph{Agrochemical}
\paragraph{Agricultural}
\paragraph{Livestock Farming}
\paragraph{Food Processing}
\paragraph{Biofuels}
\paragraph{Energy}
\paragraph{Forestry}
\paragraph{Timber}
\paragraph{Ecosystem Services}
\paragraph{Other Goods and Services}
\paragraph{Greenhouse Gas Emissions}
\paragraph{Preferences}
\paragraph{Welfare}



\subsection{Math Behind the Model}

\subsection{Model Constraints \& Resources}
\paragraph{Constraints}
FABLE is a model and as such, has certain constraints in scope and resolution. The model is not intended for detailed policymaking or land use allocation, but instead seeks an optimal distribution of land use, taking in to account the irreversibility of many land use decisions. This model cannot reflect the impacts of market failures such as imperfect information or poorly defined property rights.
\paragraph{Resources}


\end{document}